\glsresetall
\section{Introduction}
\label{sec:intro}

Having computers understand visual input is desirable in several areas.
A domestic assistant robot may use a camera to navigate and identify useful objects in a home.
Driver-less cars need to be able to read road signs and track other moving vehicles.
Web crawlers may extract information from images alongside text on the web.

This kind of understanding involves processing sensory (such as visual) input on a cognitive level.
Low-level image processing may include tasks such as prominent color extraction, edge detection and visual pattern recognition.
Higher-level processing, however, includes identifying objects, their properties and their relations to each other.
This information can then be used for language understanding, reasoning, prediction and other cognitive processes.
Making the connection between sensory input and cognitive categories is what concerns the field of \textit{perceptual semantics} \citep{PustejovskyPerceptualsemanticsconstruction1990}.

Humans use language to communicate information.
Thus it is useful to add linguistic capacities to a perceptual system.
With vision and language connected, a robot can talk about what it sees, and descriptions can be automatically generated for images found on the web.
Image caption generation is indeed a popular task for evaluating computer vision systems.
Another one is \textit{\gls{vqa}}, where the system is expected to generate answers to natural-language questions in the context of a given image.

The connection between different modes of information, such as vision and language, requires a model of semantic representation.
Formal models such as \gls{fol} have long been of choice, but recent developments have seen data-driven approaches excel in some cases.
Briefly put, the former kind tends to deliver deep structures of information in narrow domains, while the latter more easily covers wide domains, but with shallow information content \citep{SimonDobnikModularmechanisticnetworks2017}.
A recent contribution that combines several branches in formal systems is \textit{\gls{ttr}} \citep{CooperAustiniantruthattitudes2005,CooperTypetheorylanguage2016}.

The goal of this thesis is to provide a Python implementation of perception and classification of objects, recognition of spatial relations and basic natural language understanding in \gls{ttr}.
The purpose of this task is to use \gls{ttr} as a multi-modal knowledge representation system, providing formal transparency.
The implementation shall build on past proposals and focus on high-level cognition, meaning connecting vision and language through a cognitive model using \gls{ttr} as far as possible.
Lower-level parts shall use existing solutions or basic placeholder implementations.
The result shall function as a basic \gls{vqa} application, meaning it is able to answer questions given with an image as context.
The extensive use of \gls{ttr} in the application, as well as its \gls{vqa} abilities, are the points on which the result is finally qualitatively evaluated.

The theoretical background for this thesis is summarized in \autoref{sec:background}.
The work put forward is outlined in \autoref{sec:method}, and presented in its entirety in \autoref{sec:results}, including a \gls{ttr} model formulation and Python source code.
A discussion follows, before an account of conclusions in \autoref{sec:conclusions}.