\renewcommand{\sectionautorefname}{Section}
\let\subsectionautorefname\sectionautorefname
\let\subsubsectionautorefname\sectionautorefname
\section{Discussion}
\label{sec:discussion}



\subsection{Subtype check}
\label{sec:discussion-subtype}

Once the perceived scene and the parsed question had been modeled as two situation types, the task of finding an answer to the question was reduced to a subtype check.
Under the current restriction to polar questions, type theory has thus proven itself useful and straightforward as a means for the \gls{vqa} problem (and, in extension, question answering in general).

\gls{ttr} in particular posed a problem for the subtype check due to the dependence on field labels.
This feature necessitated an extra algorithmic layer to allow a label-insensitive comparison in the proposed relabel-subtype check.
As a variant of the subtype check, this solves the problem within the scope of this project, but if the model of perception is extended, the computation time grows quickly.
In particular, generating more basic-type fields would drastically increase the number of relabelings.



\subsection{PyTTR}
\label{sec:discussion-pyttr}

The PyTTR programming library provided the ability to work with \gls{ttr} types, objects and operations.
Some extensions were needed in order to realise the present project.
Some of these were quite simple, providing more or less basic operations through only a few lines of code (such as copying a record type).
These could be implemented directly in the PyTTR library.
Others provided operations that were quite specific to the use case at hand, such as ``combining'' record types with label conflict resolving and deduplicated field types.
As such, they are less suited for direct inclusion in PyTTR, and should remain in the project-specific source code.
(As both PyTTR and the source code for this project are released open-source, all parts of the implementation are open for anyone to reuse.)



\subsection{Inference-first}
\label{sec:discussion-inferencefirst}

The algorithm of perception described in \autoref{sec:agent} performs classification of spatial relations on all pairs of individuated objects.
In other words, all of the agent's beliefs are inferred at once.
Later, when attempting to answer the given question, the beliefs can be queried directly in the subtype check.

In this application, where the goal is to answer a given question, that means spending more effort than necessary.
So far as this project goes, this is no significant impediment, as the number of inferences to make (namely, classification of spatial relations) is rather small.
In extension, however, the computation time will grow with the amount of inferences (as well as the amount of detected objects) and this approach does not scale.

A more viable alternative is to first parse the question and then perform inference as needed to arrive to an answer.
If the question is about the spatial relation between a dog and a person, it will probably be enough to see that there is a dog and a person in the scene, and that the spatial relation between them matches the one expressed in the question.
