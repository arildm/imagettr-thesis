\renewcommand{\sectionautorefname}{Section}
\let\subsectionautorefname\sectionautorefname
\let\subsubsectionautorefname\sectionautorefname
\section{Discussion}
\label{sec:discussion}

%A flexible formal-semantic framework, \gls{ttr} is here being used to model meaning apprehended from different modalities: vision and natural language.

Once the perceived scene and the parsed question had been modeled as two situation types, the task of finding an answer to the question was reduced to a subtype check.
Under the current restriction to polar questions, type theory has thus proven itself useful and straightforward as a means for the \gls{vqa} problem (and, in extension, question answering in general).
\gls{ttr} in particular, however, posed a problem for the subtype check due to the dependence on field labels.
This feature necessitated an extra algorithmic layer to allow a label-insensitive comparison in the proposed relabel-subtype check.

The PyTTR programming library provided the ability to work with \gls{ttr} types, objects and operations.
Some extensions were needed in order to realise the present project.
Some of these were quite simple, providing more or less basic operations through only a few lines of code (such as copying a record type).
These could be implemented directly in the PyTTR library.
Others provided operations that were quite specific to the use case at hand, such as ``combining'' record types with label conflict resolving and deduplicated field types.
As such, they are less suited for direct inclusion in PyTTR, and should remain in the project-specific source code.
(As both PyTTR and the source code for this project are released open-source, all parts of the implementation are open for anyone to reuse.)
% The project is open source. URL?
