\section{Code}

\newcounter{cellc}
\newenvironment{cell}
  {\stepcounter{cellc}\subsection*{Cell \arabic{cellc}}\label{cell:\arabic{cellc}}\verbatim}
  {\endverbatim}

\makeatletter
\def\verbatim@font{\scriptsize\ttfamily}
\makeatother

\begin{cell}
import sys
sys.path.append('pyttr')
from pyttr.ttrtypes import *
from pyttr.utils import *
import PIL.Image
\end{cell}

\begin{cell}
Ind = BType('Ind')
\end{cell}

\begin{cell}


# Can be called with multiple args *at the end* of a code block to illustrate PyTTR types and objects.
def latex(*objs, code=False):
    # code = True # Uncomment to always get code.
    texcode = '\n\n'.join(to_ipython_latex(obj) for obj in objs)
    if code: print(texcode)
    return Latex(texcode)

# Redefine Image.show() to work with Rec.show().
def image_show(self):
    return str(self)
PIL.Image.Image.show = image_show
\end{cell}

\begin{cell}
def rectype_relabels(T, rlbs):
    """Relabel multiple fields, given a dict of from-to pairs."""
    T = T.copy()
    # Add a temporary label to each pair, to avoid overwriting.
    # This is not a guard against {a:c, b:c}, but against {a:b, b:c}.
    rlbs = [(l1, l2, gensym('rlb')) for l1, l2 in rlbs.items()]
    # First relabel all fields to tmp labels, then all to desired labels.
    for l1, l2, l_tmp in rlbs:
        T.Relabel(l1, l_tmp)
    for l1, l2, l_tmp in rlbs:
        T.Relabel(l_tmp, l2)
    return T

def is_basic_type(T):
    """Whether a type is a "basic field", i.e. a BType or a SingletonType of a BType."""
    return (isinstance(T, SingletonType) and is_basic_type(T.comps.base_type)) or isinstance(T, BType)

def basic_fields(T):
    """The labels of basic fields in a RecType."""
    return [k for k, v in T.fields() if is_basic_type(v)]

def nonbasic_fields(T):
    """The labels of non-basic fields in a RecType."""
    return [k for k, v in T.fields() if not is_basic_type(v)]
\end{cell}

\begin{cell}
T = RecType({'a': 1, 'b': 2, 'c': 3})
# Valid relabeling; is handled.
print(show(rectype_relabels(T, {'a': 'b', 'b':'d'})))
# Invalid relabeling; fails. Fix your usage.
print(show(rectype_relabels(T, {'a': 'c', 'b':'c'})))
\end{cell}

\begin{cell}
# mkptype saves the created PType objects. If called a second time with the same pred and vars,
# the existing object is returned, instead of instantiating PType every time.
ptypes = dict()
def mkptype(sym, types=[Ind], vars=['v']):
    """Make preds and ptypes identifiable by their predicate names."""
    id = '/'.join([sym, ','.join(show(type) for type in types), ','.join(vars)])
    if id not in ptypes:
        ptypes[id] = PType(Pred(sym, types), vars)
    return ptypes[id]

def create_fun(pred_name, vars=['a']):
    """Create a function of a given number of Inds (length of vars).
    
    Example: create_fun('give', 'abc') --> \a. \b. \c. give(a, b, c)
    """
    fun = mkptype(pred_name, types=[Ind]*len(vars), vars=vars)
    for v in reversed(vars):
        fun = Fun(v, Ind, fun)
    return fun

latex(create_fun('give', 'abc'))
\end{cell}

\begin{cell}
def dedupe(T):
    """Make a copy of a record type without duplicated field values."""
    for l,v in T.fields():
        # Are there more fields with the same value?
        l2s = [l2 for l2,v2 in T.fields() if l2 != l
               # Dedupe singleton types and complex types.
               and (isinstance(v, SingletonType) or not is_basic_type(v))
               # Using show is error-prone, pyttr should have an equals method.
               and show(v) == show(v2)]
        if len(l2s):
            # Relabel all these fields to the same label, overwriting until one remains.
            for l2 in l2s:
                T.Relabel(l2, l)
            # Dependent fields have changed, so start over.
            return dedupe(T)
    # No more duplicates.
    return T

T = RecType({
    'x': SingletonType(Ind, 'a'),
    'y': SingletonType(Ind, 'a'),
    'c': (create_fun('foo'), ['x']),
    'd': (create_fun('foo'), ['y']),
})
latex(T, dedupe(T.copy()))
\end{cell}

\begin{cell}
class LambdaFun(Fun):
    """Models a unary Python function as a TTR function.
    
    Type/subtype checking on a Fun will app() on a HypObj and use the result.
    In a LambdaFun, the body function may require a real object,
    and would have to be rewritten to be able to handle a HypObj for this purpose.
    To avoid this, an example argument can be specified in __init__,
    in which case that will be used in place of a HypObj.
    """
    def __init__(self,dom,body, example=None):
        self.__setattr__('domain_type',dom)
        self.__setattr__('body',body)
        self.__setattr__('example', example)
    def show(self):
        return 'lambda r' + ':' + self.domain_type.show() + ' . ' + self.body.__name__ + '(r)'
    def to_latex(self):
        return '\\lambda r' + ':' + self.domain_type.to_latex() + '\\ .\\ ' + self.body.__name__ + '(r)'
    def validate(self):
        return isinstance(self.domain_type,Type)

    def app(self,arg):
        if self.example is not None and LambdaFun.is_hypobj(arg):
            arg = self.example
        
        res = self.body(arg)
        if 'eval' in dir(res):
            return res.eval()
        else:
            return res

    def is_hypobj(r):
        return isinstance(r, HypObj) or \
            (isinstance(r, Rec) and forsome([v for l,v in r.fields()], LambdaFun.is_hypobj))

    def subst(self,v,a):
        return self

P = FunType(Ind, Ty)
T = RecType({'f': P})
def on_a(r):
    return RecType({'c': r.f.app('z')})
fun = LambdaFun(T, on_a, Rec({'f': create_fun('ex')}))
print(show(fun.app(Rec({'f': create_fun('dog')}))))

# With hypobj.
F = FunType(T, RecTy)
print(F.query(fun))
\end{cell}

\begin{cell}
# Basic types.

# Ind has already been defined.

Int = BType('Int')
Int.learn_witness_condition(lambda x: isinstance(x, int))
print(Int.query(365))

String = BType('String')
String.witness_conditions.append(lambda s: isinstance(s, str))
print(String.query('Hello World!'))

Image = BType('Image')
Image.learn_witness_condition(lambda x: isinstance(x, PIL.Image.Image) or x is None)
img = PIL.Image.open('res/dogride.jpg') # https://www.flickr.com/photos/hickatee/34017375600
print(Image.query(img))

# Segment type: a rectangular area of a given image.

Segment = RecType({'cx': Int, 'cy': Int, 'w': Int, 'h': Int})
print(Segment.query(Rec({'cx': 100, 'cy': 150, 'w': 40, 'h': 20})))
\end{cell}

\begin{cell}
PTy = Type('PType')
PTy.learn_witness_condition(lambda p: isinstance(p, PType) or (
    isinstance(p, HypObj) and forsome(p.types, lambda t: isinstance(t, PType))))

Ppty = FunType(Ind, PTy)
Obj = RecType({'seg': Segment, 'pfun': Ppty})
Objs = ListType(Obj)
ObjDetector = FunType(Image, Objs)

latex(ObjDetector)
\end{cell}

\begin{cell}
# Instantiate YOLO.

from darkflow.net.build import TFNet

tfnet = TFNet({"model": "yolo/yolo.cfg", "load": "yolo/yolo.weights",
    'config': 'yolo', "threshold": 0.2})
\end{cell}

\begin{cell}
# Function to apply YOLO to a given image.

import numpy as np

yolo_out = dict()
def yolo(img):
    """Invokes YOLO on a PIL image, caches and returns the result."""
    if str(img) not in yolo_out:
        res = tfnet.return_predict(np.array(img))
        # Save the most confident detections.
        res.sort(key=lambda o: -o['confidence'])
        yolo_out[str(img)] = res[:7]
    return yolo_out[str(img)]

def yolo_coords(o):
    """Extract the coordinates from a YOLO output item as ((x0,y0), (x1,y1))."""
    return (o['topleft']['x'], o['topleft']['y']), (o['bottomright']['x'], o['bottomright']['y'])

def xy1xy2_to_cwh(x1, y1, x2, y2):
    '''Transform to center, width and height.'''
    return {'cx': int(x1/2 + x2/2), 'cy': int(y1/2 + y2/2), 'w': x2 - x1, 'h': y2 - y1}

def yolo_reformat(o):
    """Discards the confidence item and reformats the coordinates."""
    return {'label': o['label'],
        'loc': xy1xy2_to_cwh(*sum(yolo_coords(o), ()))}
\end{cell}

\begin{cell}
from PIL import ImageFont, ImageDraw
from IPython.display import display

# Generate distinguishable colors.
phi = 2 / (1 + 5 ** .5)
colors = ('hsl({}, 90%, 70%)'.format(int(x * 360)) for x in count(0, phi))

def yolo_annotate(img):
    """Displays the image with YOLO results annotated."""
    img_annotated = img.copy()
    draw = ImageDraw.Draw(img_annotated)
    for o in yolo(img):
        color = next(colors)
        draw.rectangle(yolo_coords(o), outline=color)
        draw.text(yolo_coords(o)[0], o['label'], fill=color)
    display(img_annotated)
    
yolo_annotate(img)
# Modified image under licence CC-by-nc-sa: https://creativecommons.org/licenses/by-nc-sa/2.0/
\end{cell}

\begin{cell}
# Representing detected objects in TTR.

def yolo_detector(i):
    """Creates IndObj records for YOLO results."""
    for o in yolo(i):
        o = yolo_reformat(o)
        yield Rec({
            'seg': Rec(o['loc']),
            'pfun': create_fun(o['label'].replace(' ', '_')),
        })
ObjDetector.witness_cache.append(yolo_detector)

objs = list(yolo_detector(img))

print(ObjDetector.query(yolo_detector))
print(Objs.query(objs))
print(Obj.query(objs[0]))
print(Ppty.query(objs[0].pfun))
print(Segment.query(objs[0].seg))

latex(objs)
\end{cell}

\begin{cell}
location_ptype = mkptype('location', [Ind, Segment], ['k', 'l'])
cl_fun = Fun('k', Ind, Fun('l', Segment, location_ptype))

IndObj = RecType({
    'x' : Ind,
    'loc' : Segment,
    'cp' : PTy,
    'cl' : (cl_fun, ['x', 'loc']),
})
IndFun = FunType(Obj, RecTy)

def indfun(r):
    if not Obj.query(r):
        raise ValueError('Input must be Obj')
    indobj = RecType({
        'x': SingletonType(Ind, Ind.create()),
        'cp': (r.pfun, ['x']),
        'loc': SingletonType(Segment, r.seg),
        'cl': (cl_fun, ['x', 'loc']),
    })
    if not indobj.subtype_of(IndObj):
        raise ValueError('The result is not a subtype of IndObj')
    return indobj
IndFun.witness_cache.append(indfun)

indobjs = [indfun(r) for r in objs]

print(Obj.query(objs[1]))
print(RecTy.query(indobjs[1]))
print(indobjs[1].subtype_of(IndObj))
print(IndObj.query(indobjs[1].create_hypobj()))
latex(indobjs)
\end{cell}

\begin{cell}
from itertools import product

LocTup = RecType({'lo': IndObj, 'refo': IndObj})
RelClf = FunType(LocTup, RecTy)

location_relation_classifiers = {
    # predicate: classifier
    'left': lambda a, b: a.cx < b.cx,
    'right': lambda a, b: a.cx > b.cx,
    'above': lambda a, b: a.cy < b.cy,
    'below': lambda a, b: a.cy > b.cy,
}

def relclf(r, pred, f):
    # Support type checking of this function, which uses HypObj.
    if isinstance(r.lo.loc.cx, HypObj):
        return RecType()
    # @TODO Why not? IndObj.query(r.lo) and IndObj.query(r.refo)
    if f(r.lo.loc, r.refo.loc):
        c = create_fun(pred, 'ab')
        return RecType({
            'x': SingletonType(Ind, r.lo.x),
            'y': SingletonType(Ind, r.refo.x),
            'cr': (c, ['x', 'y']),
        })
    return RecType()

def get_relclfs():
    relclfs = []
    for pred, f in location_relation_classifiers.items():
        relclfs.append(lambda r, pred=pred, f=f: relclf(r, pred, f))
    return relclfs

relclfs = [LambdaFun(LocTup, relclf) for relclf in get_relclfs()]

loctups = []
def find_all_rels(indobjs):
    """Find all relations between IndObj records."""
    for relclf in relclfs:
        for loT, refoT in product(indobjs, indobjs):
            loctup = Rec({'lo': loT.create(), 'refo': refoT.create()})
            loctups.append(loctup)
            yield relclf.app(loctup)
        
rels = list(rel for rel in find_all_rels(indobjs) if len(rel.labels()) > 0)

latex(rels)
\end{cell}

\begin{cell}
# The classical method using nesting and flattening.

from functools import reduce

Prev = LambdaFun(RecTy, lambda old:
         LambdaFun(RecTy, lambda new:
           RecType({'prev': old}).merge(new).flatten()))

def combine_prev(Ts):
    f = lambda old, new: Prev.app(old).app(new)
    return dedupe(reduce(f, Ts, RecType()))
\end{cell}

\begin{cell}
# Custom method with unique labels before merging.

def unique_labels(T):
    """Relabel a RecType so each field label is unique over all RecTypes."""
    rlb = ((l, gensym(l)) for l in T.labels() if '_' not in l)
    return rectype_relabels(T, dict(rlb))

def combine(Ts):
    """Combine a list of belief record types into one."""
    f = lambda a, b: a.merge(unique_labels(b))
    return dedupe(reduce(f, Ts, RecType()))
\end{cell}

\begin{cell}
bel = indobjs + rels

# combine = combine_prev
bel_comb = combine(bel)

latex(bel_comb)
\end{cell}

\begin{cell}
import nltk
from nltk.sem.logic import ApplicationExpression, AndExpression, ExistsExpression, ConstantExpression

# Parsing to PyTTR cannot really be done directly. NLTK feature grammars support output in the form
# of strings or FOL, where variable substitution is only allowed in FOL. Here we produce a FOL
# expression and later translate it to a PyTTR record type.

grammar = nltk.grammar.FeatureGrammar.fromstring(r'''
%start S
S[SEM=<?np(?vp)>] -> NP[SEM=?np] VP[SEM=?vp]
S[SEM=?q] -> QS[SEM=?q]
QS[SEM=<?np(?pp)>] -> 'is' 'there' NP[SEM=?np] PP[SEM=?pp]
QS[SEM=<?np(\P.true)>] -> 'is' 'there' NP[SEM=?np]

NP[SEM=<?det(?n)>] -> Det[SEM=?det] N[SEM=?n]
Det[SEM=<\P Q.exists x.(P(x) & Q(x))>] -> 'a' | 'an'

VP[SEM=?pp] -> 'is' PP[SEM=?pp]
PP[SEM=<\x.(?np(\y.?prep(x, y)))>] -> Prep[SEM=?prep] NP[SEM=?np]

Prep[SEM=<left>] -> 'to' 'the' 'left' 'of'
Prep[SEM=<right>] -> 'to' 'the' 'right' 'of'
Prep[SEM=<above>] -> 'above'
Prep[SEM=<below>] -> 'below'
N[SEM=<aeroplane>] -> 'aeroplane'
N[SEM=<backpack>] -> 'backpack'
N[SEM=<bicycle>] -> 'bicycle'
N[SEM=<car>] -> 'car'
N[SEM=<dog>] -> 'dog'
N[SEM=<giraffe>] -> 'giraffe'
N[SEM=<kite>] -> 'kite'
N[SEM=<person>] -> 'person'
N[SEM=<snowboard>] -> 'snowboard'
N[SEM=<tent>] -> 'tent'
N[SEM=<tree>] -> 'tree'
''')
parser = nltk.FeatureChartParser(grammar)

texts = [
    'A dog is to the left of a bicycle',
    'Is there a dog to the left of a bicycle?',
    'Is there a person?',
]

def fol_to_pyttr(expr, T=RecType()):
    """Turns a FOL object into a RecType."""
    # Existential quantifier -> Ind field.
    if isinstance(expr, ExistsExpression):
        T.addfield(str(expr.variable), Ind)
        return fol_to_pyttr(expr.term, T)
    
    # Application -> ptype, e.g. left(x, y)
    if isinstance(expr, ApplicationExpression):
        pred, args = expr.uncurry()
        # Create a PType function, e.g. lambda x:Ind . dog(x)
        fun = create_fun(str(pred), 'abcd'[:len(args)])
        T.addfield(gensym('c'), (fun, [str(a) for a in args]))
        return T
    
    # For and-expressions, interpret each term.
    if isinstance(expr, AndExpression):
        fol_to_pyttr(expr.first, T)
        fol_to_pyttr(expr.second, T)
        return T
    
    # A constant function in the "is there an X" rule trivially gives "true".
    if isinstance(expr, ConstantExpression) and str(expr.variable) == 'true':
        return T
    
    raise ValueError('Unknown expression: ' + str(type(expr)) + ' ' + str(expr))

def eng_to_pyttr(text):
    # Tokenize.
    sent = text.lower().strip('.?!').split()
    # NLTK-parse to syntax tree.
    try:
        trees = parser.parse(sent)
    except ValueError:
        return RecType()
    # Extract semantic representation for the tree.
    sem = nltk.sem.root_semrep(list(trees)[0])
    # Interpret to TTR record type.
    T = fol_to_pyttr(sem, RecType())
    return T

print(texts[1])
q = eng_to_pyttr(texts[1])
latex(q)
\end{cell}

\begin{cell}
from itertools import permutations, combinations

def my_subtype_of(sub, sup):
    """Is T a subtype of U? Accept dependent rectype fields with simplistic equality test."""
    try:
        return sub.subtype_of(sup)
    except AttributeError:
        return isinstance(sub, tuple) and isinstance(sup, tuple) and show(sub) == show(sup)

def find_subtype_relabeling(S, Q):
    '''Could record type S be a sub type of record type Q if relabeling in Q is allowed?'''
    # For each relabeling of basic-type fields
    for sls in permutations(basic_fields(S), len(basic_fields(Q))):
        # Try the basic-fields relabeling of Q
        rlb_basic = dict(zip(basic_fields(Q), sls))
        Q2 = rectype_relabels(Q, rlb_basic)
        
        # For each Q field, find a S field that is a subtype
        rlb_dep = dict()
        for ql in nonbasic_fields(Q2):
            for sl in nonbasic_fields(S):
                # The new labels must be unique.
                if sl in rlb_dep.values(): continue
                if my_subtype_of(S.field(sl), Q2.field(ql)):
                    rlb_dep[ql] = sl
                    break
            if ql not in rlb_dep:
                break

        # Successful if all non-basic fields match.
        if len(rlb_dep) == len(nonbasic_fields(Q2)):
            return dict(**rlb_basic, **rlb_dep)
    return None
\end{cell}

\begin{cell}
utts_rlb = []
for text in texts:
    q = eng_to_pyttr(text)
    rlb = find_subtype_relabeling(bel_comb, q)
    print(text + ' - ' + str(bool(rlb)))
    if rlb:
        utt_rlb = rectype_relabels(q, rlb)
        utts_rlb.append(utt_rlb)
latex(utts_rlb)
\end{cell}

\begin{cell}
AgentState = RecType({
    'img': Image,
    'perc': Objs,
    'bel': ListType(RecTy),
    'utt': String,
    'que': RecTy
})
Agent = RecType({
    'objdetector': ObjDetector,
    'indfun': IndFun,
    'relclfs': ListType(RelClf),
    'state': AgentState,
})
latex(Agent)
\end{cell}

\begin{cell}
st = Rec({
    'perc': [],
    'bel': [],
    'img': None,
    'utt': '',
    'que': RecType(),
})
ag = Rec({
    'objdetector': yolo_detector,
    'indfun': indfun,
    'relclfs': relclfs,
    'state': st,
})

print(AgentState.query(st))
print(Agent.query(ag))
\end{cell}

\begin{cell}
# The combined beliefs are "cached".
bel = None

def agent_see(ag, img):
    if not Agent.query(ag): raise ValueError
    ag.state.img = img
    ag.state.perc = list(ag.objdetector(ag.state.img))
    ag.state.bel = [indfun(r) for r in ag.state.perc]
    
    for Ta, Tb in product(ag.state.bel, ag.state.bel):
        if Ta == Tb: continue
        loctup = RecType({'lo': Ta, 'refo': Tb})
        for relclf in ag.relclfs:
            new_bel = relclf.app(loctup.create_hypobj())
            if new_bel:
                ag.state.bel.append(new_bel)
                
    global bel
    bel = combine(ag.state.bel)
    if not Agent.query(ag): raise ValueError

def agent_hear(ag, text):
    if not Agent.query(ag): raise ValueError
    ag.state.utt = text
    ag.state.que = eng_to_pyttr(text)
    if not Agent.query(ag): raise ValueError

def agent_answer(ag):
    rlb = find_subtype_relabeling(bel, ag.state.que)
    return bool(rlb)

q = 'is there a person above a bicycle'
print('Q:', q)
agent_see(ag, img)
agent_hear(ag, q)
a = agent_answer(ag)
print('A:', a)
\end{cell}

\begin{cell}
vqa_items = [
    {'imgpath': 'res/vqa1.jpg', 'questions': [
        'Is there an aeroplane?',
        'Is there a person to the right of a car?',
    ]},
    {'imgpath': 'res/vqa2.jpg', 'questions': [
        'Is there a tent?',
        'Is there a kite above a person?',
    ]},
    {'imgpath': 'res/vqa3.jpg', 'questions': [
        'Is there a snowboard above a person?',
        'Is there a snowboard below a person?',
    ]},
    {'imgpath': 'res/vqa4.jpg', 'questions': [
        'Is there a giraffe below a tree?',
    ]},
]

def vqa_eval(imgpath, questions):
    img = PIL.Image.open(imgpath)
    yolo_annotate(img)
    agent_see(ag, img)
    for question in questions:
        agent_hear(ag, question)
        ans = 'Yes' if agent_answer(ag) else 'No'
        print(question + ' - ' + ans)

for item in vqa_items:
    vqa_eval(**item)
\end{cell}
