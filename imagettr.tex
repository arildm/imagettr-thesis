\documentclass[11pt, a4paper]{article}

\usepackage{mlt-thesis-2015}

\usepackage[english]{babel}
\usepackage{graphicx}
\usepackage{setspace}

\usepackage[acronym]{glossaries}
\glsdisablehyper

\title{Visual question answering with type theory}
\subtitle{Using \gls{ttr} to model visual perception and language}
\author{Arild Matsson}
\date{}

\newacronym{hmm}{HMM}{Hidden Markov Model}
\newacronym{lstm}{LSTM}{Long Short-Term Memory}
\newacronym{cnn}{CNN}{convolutional neural network}
\newacronym{ttr}{TTR}{type theory with records}
\newacronym{nlg}{NLG}{natural language generation}
\newacronym{vqa}{VQA}{visual question answering}
\newacronym{yolo}{YOLO}{You only look once}

\begin{document}

%% ============================================================================
%% Title page
%% ============================================================================
\begin{titlepage}

\maketitle

\vfill

\begingroup
\renewcommand*{\arraystretch}{1.2}
\begin{tabular}{l@{\hskip 20mm}l}
\hline
Master's Thesis: & 30 credits\\
Programme: & Master’s Programme in Language Technology\\
Level: & Advanced level \\
Semester and year: & Spring, 2018\\
Supervisors & Simon Dobnik and Staffan Larsson\\
Examiner & (name of the examiner)\\
Keywords & type theory, image recognition, object recognition, visual question answering
\end{tabular}
\endgroup

\thispagestyle{empty}
\end{titlepage}

%% ============================================================================
%% Abstract
%% ============================================================================
\newpage
\singlespacing
\section*{Abstract}

%\begin{abstract}
I'm connecting an off-the-shelf image recognition system to PyTTR, a Python implementation of TTR.
This enables representing concepts in the image as TTR types.
Then I'm implementing a \gls{vqa} task.
Questions are represented in TTR through really simple text parsing, and are answered using the representation of the image recognition results.
This shows that \gls{ttr} is helpful for representing visual perception, semantics and cognition.
\glsresetall
%\end{abstract}

\thispagestyle{empty}

%% ============================================================================
%% Preface
%% ============================================================================
\newpage
\section*{Preface}

Acknowledgements, etc.

\thispagestyle{empty}

%% ============================================================================
%% Contents
%% ============================================================================
\newpage

\begin{spacing}{0.0}
\tableofcontents
\end{spacing}

\thispagestyle{empty}

%% ============================================================================
%% Introduction
%% ============================================================================
\newpage
\setcounter{page}{1}

\section{Introduction}
\label{sec:intro}

This project will explore how \gls{ttr} can be used with image recognition tasks such as question answering and description generation.
How can off-the-shelf image recognition software output be expressed in \gls{ttr}, and what can we do with it?

\begin{itemize}
\item Express image classification results in \gls{ttr}
\item Detect and recognize multiple objects in an image, as well as relationships between them
\begin{itemize}
\item Spatial, geometric relationships such as "above" and "to the left of"
\item Interaction such as "riding" or "holding"
\end{itemize}
\item Question anwering: parsing questions or statements into \gls{ttr} and judging their validity in relation to an image
\item \gls{nlg} of image descriptions
\end{itemize}

\noindent
Development should focus on utilizing and possibly extending PyTTR, a Python implementation of \gls{ttr} \citep{pyttr}. Image recognition and natural language parsing/generation should use existing solutions as far as possible.

\section{Background}

\subsection{Type theory}

\subsubsection{Type theory and natural language}

\glsreset{ttr}
\subsubsection{\Gls{ttr}}

\gls{ttr} is a formal framework for semantics \citep{CooperRecordsRecordTypes2005}.
It has been employed to model natural language in the context of dialogue, situated agents and spoken language.

\subsubsection{Applications of \gls{ttr}}

\cite{DobnikModellinglanguageaction2012} model a robot that would move around and use laser range scanner or similar to collect points in space, group them into objects and detect spatial relations between them.
\Gls{ttr} is used throughout the model, accounting for perception and cognition.

\subsubsection{PyTTR}

\subsection{Visual object recognition}

Detectron

\subsubsection{YOLO object recognition model}

You only look once (YOLO) \citep{RedmonYouOnlyLook2015} is a neural network model that simultaneously predicts bounding boxes around objects and classifies the contained objects.

\section{\Gls{vqa} using \gls{ttr}}

\subsection{Parsing}

\section{Conclusions}

\subsection{Future work}

\bibliography{imagettr}
\end{document}