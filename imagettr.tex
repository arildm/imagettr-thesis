\documentclass[11pt, a4paper]{article}

\usepackage{mlt-thesis-2015}

\usepackage[english]{babel}
\usepackage[hidelinks]{hyperref}
\usepackage{graphicx}
\usepackage{setspace}
\usepackage{comment}
\usepackage{caption}
\usepackage{listings}
\usepackage{textgreek}
\usepackage{lscape}
\lstset{language=Python}

\usepackage[acronym]{glossaries}
\glsdisablehyper

\title{Implementing perceptual semantics in type theory with records (TTR)}
%\subtitle{An implementation of visual perception and spatial relations in \gls{ttr}}
\author{Arild Matsson}
\date{}

\newacronym{hmm}{HMM}{Hidden Markov Model}
\newacronym{lstm}{LSTM}{Long Short-Term Memory}
\newacronym{cnn}{CNN}{convolutional neural network}
\newacronym{ttr}{TTR}{type theory with records}
\newacronym{nlg}{NLG}{natural language generation}
\newacronym{vqa}{VQA}{visual question answering}
\newacronym{yolo}{YOLO}{You only look once}
\newacronym{nlp}{NLP}{natural language processing}

\begin{document}

%% ============================================================================
%% Title page
%% ============================================================================
\begin{titlepage}

\maketitle

\vfill

\begingroup
\renewcommand*{\arraystretch}{1.2}
\begin{tabular}{l@{\hskip 20mm}l}
\hline
Master's Thesis & 30 credits\\
Programme & Master’s Programme in Language Technology\\
Level & Advanced level \\
Semester and year & Spring, 2018\\
Supervisors & Simon Dobnik and Staffan Larsson\\
Examiner & (name of the examiner)\\
Keywords & type theory, image recognition, object recognition, perceptual semantics, \\
& visual question answering
\end{tabular}
\endgroup

\thispagestyle{empty}
\end{titlepage}

%% ============================================================================
%% Abstract
%% ============================================================================
\newpage
\singlespacing
\glsresetall
\section*{Abstract}

\Gls{ttr} combines several theories of semantic modeling in a single framework.
[more ttr appraisal: type checking, robustness, verifiable...]
The present work suggests a \gls{ttr} model of perception, spatial cognition and language.
Utilizing PyTTR, a Python implementation of \gls{ttr}, the model is then implemented as an executable script.
%The Python implementation includes reformatting the output of an external image recognition system and parsing natural language.
Over pure Python programming, \gls{ttr} provides a transparent formal specification, as well as advanced typing and type checking.
The implementation is evaluated in a \acrlong{vqa} task.
The results confirm the suitability of this approach.
... as well as identifying some desired additions to PyTTR.
... machine learning??

% TTR also provides a connection between perception and a wide range of semantic phenomena described in TTR, e.g. quantification, inference, modality, negation, semantic coordination,.

\thispagestyle{empty}

%% ============================================================================
%% Preface
%% ============================================================================
%\newpage
%\section*{Preface}

%Acknowledgements, etc.

%\thispagestyle{empty}

%% ============================================================================
%% Contents
%% ============================================================================
\newpage

\begin{spacing}{0.0}
\glsresetall
\tableofcontents
\end{spacing}

\thispagestyle{empty}

%% ============================================================================
%% Introduction
%% ============================================================================
\newpage
\setcounter{page}{1}

\glsresetall
\section{Introduction}
\label{sec:intro}

Having computers understand visual input is desirable in several areas.
A domestic assistant robot may use a camera to navigate and identify useful objects in a home.
Driver-less cars need to be able to read road signs and track other moving vehicles.
Web crawlers may extract information from images alongside text on the web.

This kind of understanding involves processing sensory (such as visual) input on a cognitive level.
Low-level image processing may include tasks such as prominent color extraction, edge detection and visual pattern recognition.
Higher-level processing, however, includes identifying objects, their properties and their relations to each other.
This information can then be used for language understanding, reasoning, prediction and other cognitive processes.
Making the connection between sensory input and cognitive categories is what concerns the field of \textit{perceptual semantics} \citep{PustejovskyPerceptualsemanticsconstruction1990}.

Humans use language to communicate information.
Thus it is useful to add linguistic capacities to a perceptual system.
With vision and language connected, a robot can talk about what it sees, and descriptions can be automatically generated for images found on the web.
Image caption generation is indeed a popular task for evaluating computer vision systems.
Another one is \textit{\gls{vqa}} \citep{AgrawalVQAVisualQuestion2015}, where the system is expected to generate answers to natural-language questions in the context of a given image.

The connection between different modes of information, such as vision and language, requires a model of semantic representation.
Formal models such as \gls{fol} have long been of choice, but recent developments have seen data-driven approaches excel in some cases.
Briefly put, the former kind tends to deliver deep structures of information in narrow domains, while the latter more easily covers wide domains, but with shallow information content \citep{Dobnik:2017ag}.
A recent contribution that combines several branches in formal systems is \textit{\gls{ttr}} \citep{CooperAustiniantruthattitudes2005,CooperTypetheorylanguage2016}.

\subsection{Contribution of this thesis}
\label{sec:contribution}

The main question explored in this thesis is the following: What are the advantages and problems faced when implementing a \gls{ttr} model of perception and language?

To qualify the question further, a number of requirements are taken into account.
The model shall build on past proposals for such models, mainly \cite{ttrspat} and \cite{lspc}.
It shall make extensive use of \gls{ttr}, and where it cannot, rely on either existing or new but simplistic models for self-contained subproblems.

Some limitations are also considered in order to narrow down the scope.
The setting for the model is a basic \gls{vqa} application, meaning it will be used to answer questions given with an image as context.
The language domain is restricted to polar (yes/no) questions.

%TODO Should I add a discussion on these requirements and limitations? Where?

In response to the requirements and limitations mentioned above, some subordinate questions emerge.

\begin{enumerate}
\item How well do employed tools do in this application, and where does the application require extending them?
\item In what extension can the model be covered by \gls{ttr} (and where does it need additional techniques)?
\item How well does the model perform in a \gls{vqa} application?
\item What advantages does such a model have against state of the art and other well-established approaches to:
    \begin{enumerate}
    \item \gls{vqa} solutions?
    \item other models of perception and language?
    \end{enumerate}
\end{enumerate}

In the thesis, a model is formulated and implemented, and the resulting implementation is evaluated in terms of the proposed questions.

%The purpose of this task is to use \gls{ttr} as a multi-modal knowledge representation system, providing formal transparency.

The theoretical background for this thesis is summarized in \autoref{sec:background}.
In \autoref{sec:method}, the strategies and techniques used for the implementation are described.
The implementation is then presented in \autoref{sec:results}.
In \autoref{sec:discussion}, the results are discussed in relation to the questions above.
Finally, some conclusions are made in \autoref{sec:conclusions}.


\glsresetall
\section{Background}
\label{sec:background}

This section will highlight some important pieces of the history of past research in relevant fields.

\subsection{Computational semantics}

[semantics]
Philosophers have long been interested in the study of meaning.
Frege.
[Frege?, formal semantics, Montague ...]

A well-established and largely capable formalism for expressing and operating on propositions is first-order logic (FOL).
["classical" computational semantics]
Computational accounts of meaning emerged [how].
They are/were characterized by [what].
[Montague, ...]
[FOL]
\citep{BlackburnComputationalsemantics2003}


[modern methodologies]
With recent advancements in computer science, ambitious computational-semantic theories are now in abundance.
As a competitor to formal systems, statistical methods have emerged which do well in various tasks within semantics.
They utilize the performance of modern computers and leverage the large amounts of data that are available as a product of our largely digitalized society.


[formal systems]
If statistical models of semantics do well in [empirical/data-driven/practical] tasks, formal theories of semantics [do what?]
\cite{BlackburnComputationalsemantics2003}:
FOPC for the win, but ``other approaches are both possible and interesting''.





\subsubsection{Perceptual semantics}
% ... in general, and spatial relations in particular

\cite{Garnhamunifiedtheorymeaning1989}:
Left/right/above/below/front/behind.
Basic meaning (of spatial relational terms).
Framework vertical constraint.
Gravitation. NESW.
Clark: canonical encounter? (left of cabinet, left of chair.)
(Ships and outer space.)

\cite{LoganComputationalAnalysisApprehension1996} account for the state of 
Basic–deictic(–intrinsic) relations.
Perceptual–conceptual representation.
"Computational theory of apprehension": spatial indexing -> reference frame -> ... Instead, we move into the conceptual level, both in perception and language, and evaluate validity from there.
(Drawbacks?)
They present "evidence" for their theory, does that evidence contradict the present approach?

lo, refo? (refo used bleow)

Terms of spatial relations (\textit{behind}, \textit{to the left of}, etc.) have three types of meanings \citep{Garnhamunifiedtheorymeaning1989}: basic, deictic and intrinsical.
The deictic and intrinsical meanings hold for relations between two objects.
The deictic meaning is relative to the coordinate frame of the speaker, while the intrinsical is relative to that of the reference object.
The basic meaning, introduced by \cite{Garnhamunifiedtheorymeaning1989}, is also relative to the speaker, but holds for a single object only.
[and? (it is basic in the way that... what?) example?]

\cite{RegierGroundingspatiallanguage2001a}:
Four models. AVS.
Many experiments.
Spatial term ratings influenced by: proximal \& center-of-mass orientations, grazing line, distance.

\cite{CoventryClassificationExtrageometricInfluences2004}:
Extra-geometric constraints on the meaning of spatial relational terms.
Especially functional.
Functional geometric framework.
[functional aspect, Coventry?]

\cite{HarnadSymbolGroundingProblem1990}:
Can a computer really ``understand'' concepts, that is, will it operate on grounded symbols or just the (arbitrary) symbols themselves?
Turing test.

\cite{SteelsSymbolGroundingProblem2007}:
Peirce: object, symbol, concept. Grounded with method.
Searle and other claim that the Symbol grounding problem is insolvable.
Steels concludes that it is solved.
Using experiments where a number of agents participate in a language game where they make up random words for preset concepts and manage to ``agree'' on which words to use for which concepts.

...



\subsection{Type theory in natural language processing}

Type theory was developed as an alternative to set theory, responding to paradoxes found in the latter.
With set theory widely serving as a \textit{foundation of mathematics}, the development of this new theory was [important].
Several different type theories have been created, of which Church's \textit{simply typed \textlambda-calculus} \cite{church40} and Martin-Löf's \textit{intuitionistic type theory} \citep{martinlof84} are some prominent examples.
\citep{CoquandTypeTheory2015} [too detailed?]

[use of type theory in nlp]

\cite{CooperRecordsRecordTypes2005} combines several theories from logic, semantics and linguistics in a single framework called \textbf{\acrfull{ttr}}.
 lambda calculus, phrase structure grammar, (DRT), (situation semantics) (what are the latter)?
It has been employed to model natural language in the context of dialogue, situated agents and spoken language.

[brief summary of TTR syntax?]



\subsubsection{Perceptual semantics in \gls{ttr}}

\cite{LarssonDialoguesHaveContent2011}

\cite{DobnikModellinglanguageaction2012}

\cite{lspc} model the point space perception of a mobile robot equipped with a rangefinder.
 that would move around and use laser range scanner or similar to collect points in space, group them into objects and detect spatial relations between them.
\Gls{ttr} is mainly used to model the transition from the perceptual to the cognitive domain as well as 
 throughout the model, accounting for perception and cognition.

\cite{ttrspat} develop this further, focusing especially on spatial relations.

\cite{LarssonFormalsemanticsperceptual2015}

Parsing: \cite{CooperRecordsRecordTypes2005}, \cite{RobinCooperAustiniantruthattitudes2005}, \cite{CooperTypetheorysemantics2012}, \cite{CooperTypetheorylanguage2016}



\subsubsection{PyTTR}

\cite{pyttr} is a Python implementation of \gls{ttr}.
It supports modeling records, record types, ptypes, functions and other TTR objects.
Additionally, it allows operations such as judgement, type checking and subtype checking.
As a Python library it also enables other features and peripheral procedures to be written in Python.

[the advantage of implementing a theoretical model]
PyTTR, itself being an implementation of TTR, allows, in turn, the implementation of models and theories that build on TTR.
By implementing a theoretical model as a computer program, it can ``come alive'' and be tested on real problems and data.
Another motivation for implementation is to compete with existing, less theoretically strong techniques to solve a given task, but it is possibly secondary to the former reason.

%[What has been written in PyTTR so far? nu, animat]



\subsection{Computer vision}

[computer vision]

[object recognition]

You only look once (YOLO) \citep{RedmonYouOnlyLook2015} is a neural network model that simultaneously predicts bounding boxes around objects and classifies the contained objects.

Among the most recent contributions to computer vision, Facebook's \textit{Detectron} \citep{Detectron2018} features outlining of identified objects and classification with impressive accuracy.

\begin{figure}[h]
\label{fig:dogbike_annotated}
\includegraphics[width=\textwidth]{dogbike_annotated}
\centering
\caption{Visualization of the labels and bounding boxes emitted by YOLO when given an image depicting a cyclist with a dog.}
\end{figure}



\subsubsection{\Acrfull{vqa}}

\cite{AgrawalVQAVisualQuestion2015} suggest \acrfull{vqa} as a [complete] challenge in AI-completeness.
``A VQA system takes as input an image and a free-form, open-ended, natural-language question about the image and produces a natural-language answer as the output''.
The initiative includes a dataset and a series of annual competitions since 2016.

\cite{AndreasLearningComposeNeural2016}

\section{Method}
\label{sec:method}

This thesis aims to produce a working implementation of an application with visual as well as natural-language input, using \gls{ttr}.
Such an application largely resembles those put forward in \cite{ttrspat} and \cite{lspc}, but there are some differences.
This application will feature:

\begin{enumerate}
\item Sensory input in the form of 2D images
\item Utilizing an external object recognition system
\item Detection of geometric spatial relations
\item Basic natural language understanding
\end{enumerate}

While operational functionality and the overall procedural code is written in Python, the core model is written in \gls{ttr} (realized as Python code using PyTTR).
As such, \gls{ttr} serves as a formal specification language.
The additional layer provides formal transparency and type robustness.

This section will describe some reasoning around the decisions taken, regarding the design of the \gls{ttr} model as well as its implementation.



\subsection{Object detection with YOLO}

You only look once (YOLO) \citep{RedmonYouOnlyLook2015} is a neural network model for image recognition.
It is trained using a loss function which takes detection as well as classification into account.
In other words, it simultaneously predicts bounding boxes and classifies the contained objects.
Unlike \cite{HeMaskRCNN2017} and others, it does not contain any recurrent layers.
The joint, recurrence-free model makes for a rather small network size, which in turn means a favorable evaluation speed.
However, compared to state of the art, it lags behind in accuracy.
The network is pretrained on the COCO dataset \cite{LinMicrosoftCOCOCommon2014}.

\begin{figure}[h]
\label{fig:dogbike_annotated}
\includegraphics[width=\textwidth]{dogbike_annotated}
\centering
\caption{Visualization of the labels and bounding boxes emitted by YOLO when given an image depicting a cyclist with a dog.}
\end{figure}

YOLO is written in C, using the Darknet neural network library \citep{darknet13}.
It can be used in Python with the TensorFlow machine learning library and the Darkflow library which translates a Darknet model to TensorFlow.

When invoked from Python, the return value is a collection of dict objects, each containing a label, coordinates and a confidence score, as exemplified in \autoref{lst:yolo_out}.
Results with confidence over a certain threshold are cast into \gls{ttr} records as described in \autoref{ssec:python}.
In this process, the bounding box coordinates are cast from a top-left and bottom-right tuple $\langle\langle x_1, y_1\rangle, \langle x_2, y_2\rangle\rangle$ to a center-width-height tuple $\langle x_c, y_c, w, h\rangle$, as the latter is more adequate for spatial classification.

\begin{lstlisting}[label=lst:yolo_out, caption=Example output of YOLO invocation]
[
	{
		'topleft': {'x': 354, 'y': 86},
		'bottomright': {'x': 551, 'y': 437},
		'label': 'person',
		'confidence': 0.80116189
	},
	{
		'topleft': {'x': 224, 'y': 234},
		'bottomright': {'x': 646, 'y': 476},
		'label': 'bicycle',
		'confidence': 0.85828924
	},
	...
]
\end{lstlisting}



\subsection{Objects and perception}

Our model of the perception of objects is largely based on \cite{lspc}.
First, the object detection algorithm returns a set of \textit{perceptual objects}.
Each of them is evidence that a certain location is associated with a certain property (such as being a dog), but it does not constrain any individual to this association.
Second, an \textit{individuated object} is generated for each perceptual object.
This describes the situation that there is an individual, which has the given property, at the given location.
In the \gls{ttr} implementation (presented in full in \autoref{ssec:ttrmodel}), the individuated object is a type.
%[what's so good about it being a type?] \cite{BarwiseSituationsAttitudes1981}

In \cite{lspc}, the world has the form of a 3D point space rather than a 2D image.
This necessitates different types for the perceptual input and the locations of perceived objects.
In the point space case, the $PointMap$ set type is used for the full ``world'', and any part of the world is simply a subset of it, so it is also a $PointMap$.
In our case, $Image$ is used for the full image but we use $Segment$ to refer to parts of it.



\subsection{Spatial relations}
\label{sec:method-spatrel}
% Classification algorithm non-TTR. Simplistic, compare to sophisitcated alternatives.

Our method of spatial relation classification is inspired by \cite{ttrspat} but more simplistic.
One simplification is that the reference frame is fixed.
This means we only consider the deictic meaning of spatial relation terms, and not the intrinsic.
``Left'' will mean to the left in the plane of the image, even if the reference object is turned on the side or toward the viewer.
Another simplification is the neglection of the \textit{functional} aspect of spatial relations \citep{CoventryInterplayGeometryFunction2001}.

In our model, a spatial classifier $\kappa$ takes two locations and returns a boolean result.
We have implemented spatial classifiers as Python functions.
For the purpose of this thesis, no sophisticated spatial classification has been considered.
Instead, a naive comparison between centers of bounding boxes was implemented.
This was done for the four relations ``left'', ``right'', ``above'' and ``below''.



\subsection{Language and \gls{vqa}}
\label{ssec:languagevqa}

In contrast to full \gls{vqa} systems, the model presented in this thesis will be restricted to a limited type of questions, namely polar questions on the location of one object in relation to another.
For example: ``Is the bicycle below the tree?''
A \textit{scene type} describing the visual scene is created by combining the types generated by the perceptual classification outlined above.
It serves as the context in which questions are evaluated.

The existing research on \gls{ttr}-based approaches to textual or phonetic parsing, overviewed in \autoref{ssec:ttnlp}, would surely cover the kind of utterances considered here.
However, there is currently no implementation available and ready to use, and parsing is not within the main focus of this thesis.
Therefore, the natural-language parsing implemented for this thesis is instead a simplistic one.
It uses feature structure context-free grammar (FCFG) tools available in NLTK \citep{BirdNaturalLanguageProcessing2009} to parse text into \gls{fol} expressions.
With a custom function, the \gls{fol} expressions are transformed to a TTR record type.
As an example, the question ``Is there a dog to the left of a car?'' is parsed into the type in \autoref{eq:uttex}.

\begin{equation}\label{eq:uttex}
\left[\begin{array}{rcl}
\text{x} &:& Ind\\
\text{y} &:& Ind\\
\text{c}_\text{dog} &:& \text{dog}(x)\\
\text{c}_\text{car} &:& \text{car}(y)\\
\text{c}_\text{left} &:& \text{left}(x, y)\\
\end{array}\right]\end{equation}

Giving the natural-language utterance a representation in the same formal framework as the image allows comparing them to each other.
The situation described by the question type will be true if there is a witness of that type \citep{BarwiseSituationsAttitudes1981,CooperAustiniantruthattitudes2005}.
The scene type, on the other hand, is considered true by virtue of being generated by perceptual classification.
It follows that the question type is true if it is a supertype of the scene type.

So rather than looking for a witness to the question type, we formulate the problem as subtype checking.
However, the subtype relation requires matching field labels, which will not be the case here, as labels are generated on the fly.
Thus, the condition is reformulated to allow relabeling \citep[pp. 133–135]{CooperTypetheorylanguage2016}:

If $Q$ is the type of a polar question utterance and $S$ is the type of the scene, the answer is YES if there is a relabeling $\eta$ such that $S \sqsubseteq Q_\eta$, and otherwise it is NO.

%[classification before/after question]



%\subsection{Evaluation}

%The system is tested on a few sentences for a few images.
%...


\section{Results}
\label{sec:results}

In this section, the model is defined.
First formally, in \gls{ttr}.
Then in Python, using PyTTR but also with some passages of non-PyTTR Python code.
The implementation necessitated some extensions to PyTTR which are also presented.
[The vqa application is presented? sd] [the problem is that currently there is no vqa application...]



\subsection{\Gls{ttr} model}
\label{ssec:ttrmodel}

Three basic types exist in the model.

\begin{description}
\item [$Ind$] A single individual object (or person), such as the reader or the Eiffel Tower.
\item [$Int$] An integer, such as 415.
\item [$Image$] A 2-dimensional digital image. It serves as an identifier to a set of extracted information, and its file type and actual data is not important in this thesis.
\end{description}

A $Segment$ is a record type describing a rectangular bounding box within an (implicit) image (\autoref{eq:seg}).
Its fields contain the center coordinates of the box ($cx$ and $cy$) and the width ($w$) and height ($h$) of the box.
$Ppty$ is the type of functions that can be applied to an individual and return a type (\autoref{eq:ppty}).
In our account the resulting type will be restricted to a ptype that is dependent on the individual, thus describing a property of it.

\begin{equation}\label{eq:seg}
Segment = \left[\begin{array}{rcl}
\text{cx} &:& Int\\
\text{cy} &:& Int\\
\text{w} &:& Int\\
\text{h} &:& Int
\end{array}\right]\end{equation}

\begin{equation}\label{eq:ppty}
Ppty = (Ind \rightarrow Type)\end{equation}

[PTy?]

A perceptual object is a record of the type $Obj$ (\autoref{eq:obj}).
An example record is given in \autoref{eq:objrec}.
%$Obj$ records are the result of performing \textit{object detection}.
%This fact is expressed in TTR as the function type $ObjDetector$ (\autoref{eq:objdetector}).
An object detector is a function from an image to a set of perceptual objects, as captured by the $ObjDetector$ function type (\autoref{eq:objdetector}).

[differences/similarites LSPC and others]

\begin{equation}\label{eq:obj}
Obj = \left[\begin{array}{rcl}
\text{seg} &:& Segment\\
\text{pfun} &:& Ppty \\
\end{array}\right]\end{equation}

\begin{equation}\label{eq:objrec}
obj =
\left[\begin{array}{rcl}
\text{seg} &=& \left[\begin{array}{rcl}
\text{cx} &=& 138\\
\text{w} &=& 276\\
\text{cy} &=& 654\\
\text{h} &=& 809
\end{array}\right]\\
\text{pfun} &=& \lambda v:Ind\ .\ \text{person}(v)\\
\end{array}\right] : Obj\end{equation}

\begin{equation}\label{eq:objdetector}
ObjDetector = ( Image \rightarrow [Obj] )
\end{equation}



\subsubsection{Individuation}

The perceptual object couples a property with a location, but it does not explicitly say anything about any individual object.
In \cite{lspc}, the step from the perceptual to the \textit{conceptual} domain is made by generating a record type that corresponds to a situation, namely the situation that a certain individual has a certain property and is at a certain location.
This situation record type is known as an \textit{individuated object}, and is a subtype of $IndObj$ (\autoref{eq:indobj}).
Here, $x$ is an individual and $loc$ is a location.
$cl$ specifies that $loc$ is the location of $x$, and the purpose of $cp$ is to declare a property of $x$.
$PTy$ is defined as a supertype of all ptypes (\autoref{eq:pty}).

\begin{equation}\label{eq:indobj}
IndObj = \left[\begin{array}{rcl}
\text{x} &:& Ind \\
\text{loc} &:& Segment \\
\text{cp} &:& PTy \\
\text{cl} &:& \text{location}(\text{x}, \text{loc}) \\
\end{array}\right]
\end{equation}

\begin{equation}\label{eq:pty}
PTy : Type
\end{equation}

A function for generating an $IndObj$ subtype from an $Obj$ record is known from \cite{lspc} as an \textit{individuation function}.
It is typed as $IndFun$ (\autoref{eq:indfun}).

\begin{equation}\label{eq:indfun}
IndFun = ( Obj \rightarrow RecType )
\end{equation}

The record type resulting from applying an $IndFun$ function should be a subtype of $IndObj$.

For each record type returned by the individuation function, a record is simultaneously created.
The $loc$ value of this record is naturally identical to the $seg$ value of the $Obj$ input record.
Objects for the remaining fields need to be instantiated on the spot.
Object creation is notated here as $A_{new}$, where the symbol $A$ may vary for the sake of readability.
%For the $x$ field, we create a new individual object $a_{new} : Ind$.
%For the ptype fields $cp$ and $cl$, we also create new objects $e_{new} : r.\text{pfun}(\text{x})$ and $e_{new} : \text{location}(\text{x}, \text{loc})$.

The individuation function is defined in \autoref{eq:indfundef}, with an example application in \autoref{eq:indfunrec}.
The definition uses manifest fields to denote the \textit{fully specified} record type, or singleton record type.

\begin{equation}\label{eq:indfundef}
f_{IndFun} = \lambda r : Obj\ . \left[\begin{array}{lcl}
    \text{x} = a_{new} &:& Ind \\
    \text{cp} = e_{new_1} &:& r.\text{pfun}(\text{x}) \\
    \text{cl} = e_{new_2} &:& \text{location}(\text{x}, \text{loc}) \\
    \text{loc} = r.\text{seg} &:& Segment\\
\end{array}\right]
\end{equation}

\begin{equation}\label{eq:indfunrec}
f_{IndFun}(
\left[\begin{array}{rcl}
\text{seg} &=& \left[\begin{array}{rcl}
\text{cx} &=& 138\\
\text{w} &=& 276\\
\text{cy} &=& 654\\
\text{h} &=& 809
\end{array}\right]\\
\text{pfun} &=& \lambda v:Ind\ .\ \text{person}(v)\\
\end{array}\right]
) =
\left[\begin{array}{lcl}
    \text{x} = a_0 &:& Ind \\
    \text{cp} = e_0 &:& \text{person}(\text{x}) \\
    \text{cl} = e_1 &:& \text{location}(\text{x}, \text{loc}) \\
    \text{loc} = \left[\begin{array}{rcl}
		\text{cx} &=& 138\\
		\text{w} &=& 276\\
		\text{cy} &=& 654\\
		\text{h} &=& 809
		\end{array}\right] &:& Segment\\
\end{array}\right]
\end{equation}



\subsubsection{Spatial relations}

Relations may hold between pairs of individuated objects.
How do we detect and model a certain relation between such a pair?

Since we are interested in the spatial relation between a \textit{reference object} and a \textit{located object}, we will be constructing tuple-like records of the type $LocTup$ defined in \autoref{eq:loctup}.
Records of this type contain instantiations (records) of two $IndObj$ record types.
In \autoref{eq:clf}, a classifier is modeled as a function from such a record to a new record type which should describe the relation.

\begin{equation}\label{eq:loctup}
LocTup = \left[\begin{array}{rcl}
    \text{lo} &:& IndObj \\
    \text{refo} &:& IndObj \\
    \end{array}\right]
\end{equation}

\begin{equation}\label{eq:clf}
ClfFun = ( LocTup \rightarrow RecType )
\end{equation}

For instance, a classifier for ``left'' might look like in \autoref{eq:leftclfdef}, where $\kappa_{left}$ is a non-TTR, boolean function.
Of course, the requirement that the individual $r.\text{lo}.\text{x}$ is actually located at $r.\text{lo}.\text{loc}$ (and same for $r.\text{refo}$) is implicit from the typing as $IndObj$, where the field $\text{cl} : \text{location}(\text{x}, \text{loc})$ is necessarily present.

\begin{equation}\label{eq:leftclfdef}
\lambda r : LocTup \ .\ 
\begin{cases}
\left[\begin{array}{rcl}
    \text{cr} &:& \text{left}(r.\text{lo}.\text{x}, r.\text{refo}.\text{x}) \\
\end{array}\right],
& \text{if } \kappa_{left}(r.\text{lo}.\text{loc}, r.\text{refo}.\text{loc}) \\
[], & \text{otherwise}
\end{cases}
\end{equation}



\subsubsection{Combining beliefs}

The hitherto generated types are considered our beliefs.
As described in \autoref{ssec:languagevqa}, we can answer to a polar question if we \textit{combine} the beliefs to a scene type and check whether it is a subtype of the question type, $S \sqsubseteq Q$.
How is this combining carried out?
A direct application of the \textit{merge} operation is not suitable for this, as we have labels reocurring in multiple record types.
Instead, all record types are relabeled with new, unique labels, before being merged.

\label{def:cfmerge}
If $T_1$ and $T_2$ are record types, then the \textbf{conflict-free merge} $T_1 \underset{u}{\wedge} T_2$ is a merge of the record types relabeled such that they do not share any labels.

If the belief types are $T_1, T_2, ..., T_n$, they are combined to $T_1 \underset{u}{\wedge} T_2 \underset{u}{\wedge} ... \underset{u}{\wedge} T_n$.



\subsubsection{Agent}

We are now connecting the perceptual-conceptuel pieces described above, by building an agent.
It receives information on classified and located objects of an image, and apprehends their basic status and spatial relations.
It also receives the result of parsing a natural-language utterance.
The information is in \gls{ttr} form, which finally allows the agent to connect the two modes.
This provides a means to answer to natural-language questions about the image.

\begin{equation}\label{eq:agent}
Agent = \left[\begin{array}{rcl}
    \text{objdetector} &:& ObjDetector \\
    \text{indfun} &:& IndFun \\
    \text{appr} &:& [(Rec \rightarrow RecType)] \\
    \text{state} &:& AgentState \\
    \end{array}\right]
\end{equation}

\begin{equation}\label{eq:state}
AgentState = \left[\begin{array}{rcl}
    \text{img} &:& Image \\
    \text{perc} &:& [Obj] \\
    \text{bel} &:& [RecType] \\
    \text{utt} &:& RecType \\
    \end{array}\right]
\end{equation}

The fields $objdetector$, $indfun$ and $appr$ of $Agent$ are to be statically defined for a specific agent.
While running, the agent will modify the $AgentState$ record in $state$.

For an agent $agt : Agent$, the perception and question-answering procedure is carried out as follows.

\begin{enumerate}
\item Visual input in the form of an image is received and assigned to $agt.\text{state}.\text{img}$.
\item $objdetector$ is invoked on $agt.\text{state.img}$ and creates a collection of records that are assigned to $agt.\text{state}.\text{perc}$.
\item $indfun$ is, in turn, invoked on each record in $agt.\text{state.perc}$ and resulting record types are added to $agt.\text{state.bel}$.
\item Now, each function in $agt.\text{appr}$ are applied:
	\begin{enumerate}
	\item The fields of the domain record type of the function is considered its arguments.
	\item Each combination of $agt.\text{state.bel}$ record types that matches the argument types is considered for input.
	\item A record is instantiated for each input record type, and the records are combined into one that matches the domain record type of the function.
	\item Resulting record types are added to $agt.\text{state.bel}$
	\end{enumerate}
	For example, the \textit{left} classifier in \autoref{eq:leftclfdef} is applied to each pair of $IndObj$ after instantiating and combining records into a $LocTup$.
\item Any language input is parsed and the resulting record type assigned to $agt.\text{state.utt}$.
\item The record types in $agt.\text{state.bel}$ are combined. If the resulting record type is a relabel-subtype of $agt.\text{state.utt}$, the answer ``yes'' is emitted; otherwise ``no''.
\end{enumerate}

An example state of an agent $agt$ is shown in \autoref{eq:agt}.

\begin{landscape}
\begin{equation}\label{eq:agt}
\renewcommand{\arraystretch}{1.2}
agt = \left[\begin{array}{rcl}
    \text{objdetector} &=& \mathtt{yolo\_detector} \\
    \text{indfun} &=& \mathtt{indfund} \\
    \text{appr} &=& [Clf_{left}, Clf_{right}, Clf_{above}, Clf_{below}] \\
    \text{state} &=& \left[\begin{array}{rcl}
		\text{img} &=& \mathtt{dogride.jpg} \\
		\text{perc} &=& [
			\left[\begin{array}{rcl}
				\text{seg} &=& \left[\begin{array}{rcl}
					\text{w} &=& 197\\
					\text{cx} &=& 452\\
					\text{h} &=& 351\\
					\text{cy} &=& 261
					\end{array}\right]\\
				\text{pfun} &=& \lambda a:Ind\ .\ \text{person}(a)
				\end{array}\right],
			\left[\begin{array}{rcl}
				\text{seg} &=& \left[\begin{array}{rcl}
					\text{w} &=& 422\\
					\text{cx} &=& 435\\
					\text{h} &=& 242\\
					\text{cy} &=& 355
					\end{array}\right]\\
				\text{pfun} &=& \lambda a:Ind\ .\ \text{bicycle}(a)
				\end{array}\right],
			...
			] \\
		\text{bel} &=& \begin{array}{l} [
			\left[\begin{array}{rcl}
				\text{x} = a_0 &:& Ind\\
				\text{cp} = e_0 &:& \text{person}(x)\\
				\text{cl} = e_{1} &:& \text{location}(x, loc)\\
				\text{loc} = \left[\begin{array}{rcl}
					\text{w} &=& 197\\
					\text{cx} &=& 452\\
					\text{h} &=& 351\\
					\text{cy} &=& 261
					\end{array}\right]
					&:& Segment \\
				\end{array}\right],
			{}{} \left[\begin{array}{rcl}
				\text{cr}=e_6 &:& \text{above}(a_{0}, a_{1})
				\end{array}\right],
			... ]
			\end{array} \\
		\text{utt} &=& \left[\begin{array}{rcl}
			\text{x} &:& Ind\\
			\text{y} &:& Ind\\
			\text{c}_\text{0} &:& \text{dog}(x)\\
			\text{c}_\text{1} &:& \text{bicycle}(y)\\
			\text{c}_\text{2} &:& \text{left}(x, y)\\
			\end{array}\right] \\
		\end{array}\right] \\
    \end{array}\right]
\end{equation}
\end{landscape}



\subsection{Python implementation}
\label{ssec:python}

This section presents significant parts of the Python implementation of the model described above.
The full code, including visualization and more comments, is published as a Jupyter Notebook at \url{https://github.com/arildm/imagettr}.



\subsubsection{PyTTR definitions}

\begin{lstlisting}[label={lst:pyttrbasic}, caption=TTR type definitions]
Ind = BType('Ind')

Int = BType('Int')
Int.learn_witness_condition(lambda x: isinstance(x, int))

Image = BType('Image')
Image.learn_witness_condition(lambda x: isinstance(x, PIL.Image.Image))

Segment = RecType({'cx': Int, 'cy': Int, 'w': Int, 'h': Int})
Ppty = FunType(Ind, Ty)
Obj = RecType({'seg': Segment, 'pfun': Ppty})
Objs = ListType(Obj)
ObjDetector = FunType(Image, Objs)
ObjDetector.witness_cache.append(yolo_detector)

PTy = Type('PTy')
PTy.learn_witness_condition(lambda p: isinstance(p, HypObj) \
    and forsome(p.types, lambda t: isinstance(t, PType)))

IndObj = RecType({
    'x' : Ind,
    'loc' : Segment,
    'cp' : PTy,
    'cl' : create_fun('location', 'ab').app('x').app('loc'),
})
IndFun = FunType(Obj, RecTy)

LocTup = RecType({'lo': IndObj, 'refo': IndObj})
ClfRes = RecType({'cr': PTy})
RelClf = FunType(LocTup, ClfRes)
\end{lstlisting}



\subsubsection{Application procedure}

No implementation has been done of the agent described above.
Instead, the procedure for object detection, spatial classification and question answering is performed by the lines of code in \autoref{lst:procedure}.
The functions used here are partially described in the following subsections.

\begin{lstlisting}[label=lst:procedure, caption=Application procedure]
objs = list(yolo_detector(img))
indobjs = [indfun(r) for r in objs]
rels = list(find_all_rels(indobjs))
bel = indobjs + rels
q = eng_to_pyttr(text)
ans = bool(find_subtype_relabeling(combine_beliefs(bel), q))
\end{lstlisting}



\subsubsection{Outside PyTTR}

\paragraph{Individuation function}

The PyTTR implementation of \gls{ttr} functions does not support usage of the argument within the function body.
The individuation function was therefore implemented in Python.
To keep the connection to \gls{ttr} tight, the argument and the result are checked against their \gls{ttr} types.

\begin{lstlisting}[label={lst:indfun},caption={Individuation function}]
def indfun(r):
    if not Obj.query(r):
        raise ValueError()
    cp = r.pfun.app('x')
    cl = create_fun('location', 'ab').app('x').app('loc')
    indobj = RecType({
        'x': SingletonType(Ind, Ind.create()),
        'cp': SingletonType(cp, cp.create()),
        'loc': SingletonType(Segment, r.seg),
        'cl': SingletonType(cl, cl.create()),
    })
    if not unsingleton(indobj).subtype_of(IndObj):
        raise ValueError()
    return indobj
IndFun.witness_cache.append(indfun)

indobjs = [indfun(r) for r in objs]
\end{lstlisting}

\paragraph{Spatial relation classification}

In \autoref{lst:relclf}, each pair of $IndObj$ types is tested for each spatial relation classifier.
A classifier is a tuple of a predicate identifier (e.g. {\tt left}) and a function from two $Segment$s to a boolean value.
The return values of {\tt get\_relclfs} are functions of the same kind as the example in \autoref{eq:leftclfdef}.

\begin{lstlisting}[label=lst:relclf, caption=Spatial relation classifiers]
def get_relclfs():
    for pred, f in location_relation_classifiers.items():
        def relclf(r):
            if f(r.lo.loc, r.refo.loc):
                c = create_fun(pred, 'ab').app(r.lo.x).app(r.refo.x)
                return RecType({'cr': SingletonType(c, c.create())})
            return RecType()
        RelClf.witness_cache.append(relclf)
        yield relclf

def find_all_rels(indobjs):
    """Find all relations between IndObj records."""
    for relclf in get_relclfs():
        for loT, refoT in product(indobjs, indobjs):
            loctup = Rec({'lo': loT.create(), 'refo': refoT.create()})
            yield relclf(loctup)
\end{lstlisting}

\paragraph{YOLO}

In \autoref{lst:yolo}, YOLO object detection is invoked on an image, and the result is converted to PyTTR records.

\begin{lstlisting}[label=lst:yolo, caption=YOLO usage]
def yolo_detector(i):
    """Creates IndObj records for YOLO results."""
    for o in yolo(i):
        o = yolo_reformat(o)
        yield Rec({
            'seg': Rec(o['loc']),
            'pfun': create_fun(o['label'].replace(' ', '_')),
        })
\end{lstlisting}



\subsubsection{Extensions to PyTTR}

The combining of beliefs and the subtype relabeling require TTR operations that are not defined in PyTTR.

\paragraph{Combining beliefs}

The conflict-free merge operation defined in \autoref{def:cfmerge} is implemented in \autoref{lst:mergeunconflict}.
For example, if $T_1 = T_2 = [x:Ind]$, then $\mathtt{merge\_unconflict}(T_1, T_2)$ evaluates to $\left[\begin{array}{rcl} x_0&:&Ind \\ x_1&:&Ind \end{array}\right]$ (or a similar record type with different subscript indices).

\begin{lstlisting}[label={lst:mergeunconflict},caption={merge\_unconflict}]
def unique_labels(T):
    """Relabel a RecType so each field label is unique over all RecTypes."""
    for l, v in T.comps.__dict__.items():
        if '_' not in l:
            T.Relabel(l, gensym(l))
    return T

def merge_unconflict(T1, T2):
    """Merge two RecTypes after making sure they do not share any field labels."""
    T1c = unique_labels(copy_rectype(T1))
    T2c = unique_labels(copy_rectype(T2))
    return T1c.merge(T2c)
\end{lstlisting}

When it comes to dependent types, a custom in \gls{ttr} literature is to include the dependum as a field and use the field label in the dependent type.
In our implementation, spatial classifiers return record types where dependums are instead specified as individuals directly.
The function in \autoref{lst:useindfieldlabels} helps when those record types are combined with others.

\begin{lstlisting}[label={lst:useindfieldlabels},caption={use\_ind\_field\_labels}]
def use_ind_field_labels(T):
    """c:foo(a) becomes c:foo(x) if x=a:Ind is present."""
    T = copy_rectype(T)
    for l, v in T.comps.__dict__.items():
        if isinstance(v, SingletonType) and v.comps.base_type == Ind:
            a = v.comps.obj
            T.Relabel(a, l)
            # Undo the relabeling of the Ind field itself.
            T.comps.__dict__[l] = SingletonType(Ind, a)
    return T
\end{lstlisting}

\autoref{lst:combine} defines the function for combining record types in the fashion required: conflict-free merging and subsequent dependum re-referencing.

\begin{lstlisting}[label={lst:combine},caption={combine}]
def combine_beliefs(bel):
    """Combine a list of belief record types into one."""
    return unsingleton(use_ind_field_labels(reduce(merge_unconflict, bel, RecType())))
\end{lstlisting}

\paragraph{Subtype-relabeling}

The condition described in \autoref{ssec:languagevqa} necessitated the ability to find a relabeling $\eta$ that would fulfill a subtype relation $S \sqsubseteq Q_\eta$.
A simple approach to finding such a relabeling is to list every possible combination and one-by-one perform relabeling and check subtypeness.
However, that is computationally very expensive, and not very interesting even if speed is not a priority.

Another quicker approach is to not perform the relabeling in all steps.
Subtypeness can instead be checked field-wise:
For every field $\langle\ell_Q, T_Q\rangle$ in $Q$, if there is a field $\langle\ell_S, T_S\rangle$ in $S$ such that $T_S \sqsubseteq T_Q$, then let $\eta(\ell_Q) = \ell_S$.
If $\eta$ covers all fields in $Q$, then it follows that $S \sqsubseteq Q_\eta$.

However, we need to mind the types that are dependent on other fields.
For example, $\text{dog}(\text{x}) \cancel{\sqsubseteq} \text{dog}(\text{y})$ even if both $\text{x}:Ind$ and $\text{y}:Ind$ are fields in the respective record types.
To remedy this, the simple approach mentioned above is carried out as a first step, but for basic-type fields only.
Then, for every basic-field relabeling, the second approach is carried out.
In the example case, some of the basic-type relabelings will then have relabeled $\text{y}$ to $\text{x}$ so that the field-wise check becomes $\text{dog}(\text{x}) \sqsubseteq \text{dog}(\text{x})$.

\begin{lstlisting}[label=lst:subtyperlb, caption=Subtype-relabeling.]
from itertools import permutations, combinations

def find_subtype_relabeling(T, U):
    '''Could record type T be a sub type of record type U if relabeling in T is allowed?'''
    # Find possible relabelings for basic-type fields
    basic_label_permutations = set(ps[:len(basic_fields(U))] for ps in permutations(basic_fields(T)))
    
    for tks in basic_label_permutations:
        # Copy U and try a basic-fields relabeling
        U2 = copy_rectype(U)
        rlb = dict(zip(basic_fields(U), tks))
        rectype_relabels(U2, rlb)
        
        # For each U field, find a T field that is a subtype
        match = dict()
        for uk in nonbasic_fields(U2):
            for tk in nonbasic_fields(T):
                if T.comps.__dict__[tk].subtype_of(U2.comps.__dict__[uk]):
                    match[uk] = tk
                    break
            if uk not in match:
                break

        # Successful if all non-basic fields match.
        if len(match) == len(nonbasic_fields(U2)):
            return dict(**rlb, **match)
    return None
\end{lstlisting}



\subsubsection{Language parsing}

Parsing to PyTTR cannot be done directly.
The output of NLTK feature grammars is either strings or \gls{fol} expressions.
For the need of variable substitution, we choose the latter.
The resulting \gls{fol} conjunction expression is then translated to PyTTR ptypes, and $Ind$ fields are created from the arguments of the predicates.

\begin{lstlisting}[label=lst:grammar, caption=Basic parsing of natural language into PyTTR object]
import nltk

grammar = nltk.grammar.FeatureGrammar.fromstring(r'''
%start S
S[SEM=<?s(x) & ?vp(x, y)>] -> NP[SEM=?s] VP[SEM=?vp]
S[SEM=?q] -> QS[SEM=?q]
QS[SEM=<?s(x) & ?vp(x, y)>] -> 'is' 'there' NP[SEM=?s] PP[SEM=?vp]
NP[SEM=<?det(?n)>] -> Det[SEM=?det] N[SEM=?n]
Det[SEM=<\P a.P(a)>] -> 'a' | 'an'
N[SEM=<dog>] -> 'dog'
N[SEM=<car>] -> 'car'
N[SEM=<person>] -> 'person'
N[SEM=<bicycle>] -> 'bicycle'
N[SEM=<backpack>] -> 'backpack'
VP[SEM=?pp] -> 'is' PP[SEM=?pp]
PP[SEM=<\a b.(?prep(a, b) & ?o(b))>] -> Prep[SEM=?prep] NP[SEM=?o]
Prep[SEM=<left>] -> 'to' 'the' 'left' 'of'
Prep[SEM=<right>] -> 'to' 'the' 'right' 'of'
Prep[SEM=<above>] -> 'above'
Prep[SEM=<under>] -> 'under'
''')
parser = nltk.FeatureChartParser(grammar)

def fopc_to_pyttr(expr, T=RecType()):
    """Turns a FOPC object into a RecType."""
    from nltk.sem.logic import ApplicationExpression, AndExpression
    if isinstance(expr, ApplicationExpression):
        pred, args = expr.uncurry()
        T.addfield(gensym('c'), mkptype(str(pred), vars=[str(a) for a in args]))
        for x in args:
            if str(x) not in T.comps.__dict__:
                T.addfield(str(x), Ind)
    if isinstance(expr, AndExpression):
        fopc_to_pyttr(expr.first, T)
        fopc_to_pyttr(expr.second, T)
    return T

def eng_to_pyttr(text):
    trees = parser.parse(text.lower().strip('.?!').split())
    sem = nltk.sem.root_semrep(list(trees)[0])
    T = fopc_to_pyttr(sem)
    return T
\end{lstlisting}



\subsection{Discussion}
\label{sec:discussion}

A \gls{ttr} model was developed, combining perception and language in a \gls{vqa} setting.
The model was then implemented in Python using PyTTR.
External systems and basic implementations were used for object detection, spatial classification and natural-language parsing.
The semantic representation framework, constituted by the \gls{ttr} model, then functions as a bridge between the perceptual systems.
With this, the aims formulated in the introduction were reached.

% Model

As discussed in \autoref{sec:method-spatrel}, our treatment of spatial relations excludes the intrinsical meaning and functional aspects, in favor of model simplicity and easy implementation.
...

The model is restricted to a limited type of polar questions.
Questions such as ``Is the dog brown?'' or ``What is to the left of the car?''
Question types.
Not a full VQA solution.
It can only answer one question type.
With only the extension of parsing, it could understand (= educe situation record types) more complex forms like "A R1 B1 and R2 B2", "A1 R1 B1 and A2 R2 B2". "A is between B1 and B2"
"What is R B?"
"What color is the A?"
"How many A are there?" etc.

%The agent is vaguely formulated?

% Implementation


The implementation lacks coverage of the enclosing agent structure.
...



%Contradiction of Logan \& Sadler's "evidence" for their "theory of apprehension" (which is different from mine)? (Already Regier \& Carlson did.)
%[no difference -sd]


\section{Conclusions}
\label{sec:conclusions}

We have implemented \gls{vqa} in PyTTR.
This involves connecting visual perception and language with formal semantic representations.
In this, TTR is the single framework that serves to represent all parts of the pipeline: perception, language and grounding.

The system utilizes an external, high-performance object recognition system.
Spatial relation detection and language parsing are instead performed with makeshift solutions.
They all share the quality of being easily replaceable by other solutions as needed.

Our model is a framework for detecting objects and relations in visual data, and grounding language in the detected information.
It provides a \gls{vqa} system based on formal semantics.

The model is one of few applications of the recently developed PyTTR library, which allows TTR to be used in executable programs.
[extension with unsingleton, relabel-subtype. also check changes in pyttr itself (relabel(), ?)]

%- What we have contributed to (1) semantic representations of grounded language in TTR; (ii) their implementation in pyTTR; (iii) in the domain of visual question answering? 

%It works and it's nice because...
robustness, type checking, verifiable (?)



\subsection{Future work}

%(b) left for future work in terms of (1) semantic representations of grounded language in TTR; (ii) their implementation in pyTTR; (iii) in the domain of visual question answering? 

Spatial templates \& regions of acceptability. Compound relations (above right) finer (directly). Functional aspect.  \cite{LoganComputationalAnalysisApprehension1996} (also Dobnik etc)

Basic->Deictic->Intrinsic relations  \cite{LoganComputationalAnalysisApprehension1996}.

4 question types.
Moar question types (tasks/programs/routines in L\&S).

%Dialogue.

Classification after Q.

Probabilistic TTR for ``good fit'' (is is more to the left or more above?).

Negation.


\bibliography{imagettr}
\end{document}