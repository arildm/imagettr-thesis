\section{Conclusions}
\label{sec:conclusions}

We have implemented the foundations of \gls{vqa} in PyTTR.
This involves connecting visual perception and language with formal semantic representations.
In this, TTR is the single framework that serves to represent all parts of the pipeline: perception, language and grounding.

The system utilizes an external, high-performance object recognition system.
Spatial relation detection and language parsing are instead performed with makeshift solutions.
They all share the quality of being easily replaceable by other solutions as needed.

Our model is a framework for detecting objects and relations in visual data, and grounding language in the detected information.
It provides a \gls{vqa} system based on formal semantics.

The model is one of few applications of the recently developed PyTTR library, which allows TTR to be used in executable programs.
\gls{ttr} operations such as combining beliefs and subtype relabeling were implemented, and the source code is released freely so that it can be reused by anyone if desired.



\subsection{Future work}

\cite{ttrspat} features \gls{ttr} accounts of the intrinsical meaning and the functional aspect of spatial relations, but they have hitherto not been implemented.

Spatial classification and language parsing were achieved using minimal and simplistic implementations.
Substituting them with sophisticated systems would make for wider question coverage and higher question answering scores.
For instance, \cite{LoganComputationalAnalysisApprehension1996} proposes spatial templates, regions of acceptability and compound relations (like ``above to the right of'').

There are still some features of \gls{ttr} not implemented in PyTTR.
One that was relevant in this project was the ability to reference an argument in a function ({\tt Fun}) body.
This lack required non-PyTTR Python usage where PyTTR should suffice.

%(b) left for future work in terms of (1) semantic representations of grounded language in TTR; (ii) their implementation in pyTTR; (iii) in the domain of visual question answering? 

%Classification after Q.
