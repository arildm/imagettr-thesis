\section{Conclusions}
\label{sec:conclusions}

We have implemented \gls{vqa} in PyTTR.
This involves connecting visual perception and language with formal semantic representations.
In this, TTR is the single framework that serves to represent all parts of the pipeline: perception, language and grounding.

The system utilizes an external, high-performance object recognition system.
Spatial relation detection and language parsing are instead performed with makeshift solutions.
They all share the quality of being easily replaceable by other solutions as needed.

Our model is a framework for detecting objects and relations in visual data, and grounding language in the detected information.
It provides a \gls{vqa} system based on formal semantics.

The model is one of few applications of the recently developed PyTTR library, which allows TTR to be used in executable programs.
[extension with unsingleton, relabel-subtype. also check changes in pyttr itself (relabel(), ?)]

%- What we have contributed to (1) semantic representations of grounded language in TTR; (ii) their implementation in pyTTR; (iii) in the domain of visual question answering? 

%It works and it's nice because...
robustness, type checking, verifiable (?)



\subsection{Future work}

%(b) left for future work in terms of (1) semantic representations of grounded language in TTR; (ii) their implementation in pyTTR; (iii) in the domain of visual question answering? 

Spatial templates \& regions of acceptability. Compound relations (above right) finer (directly). Functional aspect.  \cite{LoganComputationalAnalysisApprehension1996} (also Dobnik etc)

Basic->Deictic->Intrinsic relations  \cite{LoganComputationalAnalysisApprehension1996}.

4 question types.
Moar question types (tasks/programs/routines in L\&S).

%Dialogue.

Classification after Q.

Probabilistic TTR for ``good fit'' (is is more to the left or more above?).

Negation.
